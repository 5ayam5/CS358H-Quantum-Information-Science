\begin{solution}[label=ques:2a]
  \begin{question}
    Prove the following identity.\par
    \begin{minipage}[t]{\textwidth}
      \centering
      \begin{quantikz}
        \qw{} & \gate{H} & \gate{X} & \gate{H} & \gate[nwires=1,style={draw=none,fill=none}]{=} & \gate{Z} & \qw{}
      \end{quantikz}
    \end{minipage}

    Show that this also implies\par
    \begin{minipage}[t]{\textwidth}
      \centering
      \begin{quantikz}
        \qw{} & \gate{H} & \gate{Z} & \gate{H} & \gate[nwires=1,style={draw=none,fill=none}]{=} & \gate{X} & \qw{}
      \end{quantikz}
    \end{minipage}
  \end{question}
  \tcblower{}
  \begin{proof}
    To prove the identitiy, it is enough the prove that it holds for both the basis states $\ket{0}$ and $\ket{1}$. Note that the eigenvectors of the $X$ gate are the $\ket{+}$ and $\ket{-}$ states with the eigenvalues equal to $+1, -1$ respectively. Therefore, the result of applying the circuit on the left, $HXH$ is equal to doing nothing on the $\ket{0}$ state, whereas it adds a global phase of $-1$ on the $\ket{1}$ state as input. This is exactly what the $Z$ gate does. Therefore, the identity holds, since for any arbitrary quantum state, we will be applying the operations in superposition on both the basis states.\par
    Similarly, it is easy to see that the second identity also holds. We just pre-multiply and post-multiply both sides with the $H$ gate. This cancels out the $H$ gate on the LHS (since it is it's own inverse), resulting in $X$, but the RHS transforms to $HZH$.
  \end{proof}
\end{solution}

\begin{solution}[label=ques:2b]
  \begin{question}
    The 2-qubit CSIGN gate (also known as a controlled-Z gate) operates by  applying a relative phase shift of $-1$ to the $\ket{1}$ component of the second qubit if the first qubit is equal to 1 and otherwise does nothing. As a matrix it is given explicitly by the diagonal matrix diag(1,1,1,-1). Using part a, show how to simulate a CSIGN gate using only CNOT and Hadamard gates by writing down the appropriate circuit; show your derivation or give a brief explanation of why the circuits are equivalent.
  \end{question}
  \tcblower{}
  \begin{proof}[Solution]
    Note that the CNOT gate is equivalent to applying a conditional $X$ gate on the second qubit, conditioned on the value of the first qubit. Therefore, we can execute the CSIGN gate as follows,

    \begin{minipage}[t]{\textwidth}
      \centering
      \begin{quantikz}
        \lstick{$\ket{\psi_1}$} & & \ctrl{1} & \qw{} & \qw{}\\
        \lstick{$\ket{\psi_2}$} & \gate{H} & \targ{} & \gate{H} & \qw{}
      \end{quantikz}
    \end{minipage}

    The reason this circuit works is because the CNOT gate has no effect when the control qubit is in the $\ket{0}$ state, and therefore the $H$ gates cancel out. However, when the control qubit is in the $\ket{1}$ state, the operations on the second qubit act as if we are applying $HXH$, which is equal to $Z$, as shown in Question~\ref{ques:2a}.
  \end{proof}
\end{solution}

\begin{solution}[label=ques:2c]
  \begin{question}
    Prove the following identity:\par
    \begin{minipage}[b]{0.5\textwidth}
      \raggedleft
      \begin{quantikz}
        \qw{} & \gate{H} & \ctrl{1} & \gate{H} & \qw{}\\
        \qw{} & \gate{H} & \targ{} & \gate{H} & \qw{}
      \end{quantikz}
    \end{minipage}
    $=$
    \begin{minipage}[b]{0.5\textwidth}
      \begin{quantikz}
        \ghost{H} & \targ{} & \qw{}\\
        \ghost{H} & \ctrl{-1} & \qw{}
      \end{quantikz}
    \end{minipage}

    In other words: show that a CNOT by which qubit A controls qubit B, when viewed in a different basis, is actually a CNOT by which qubit B controls qubit A! 
This illustrates how, with quantum information, unlike with classical information, there's no way for one system to affect another one without the possibility of being affected itself.

We do not want you to solve this problem by brute force. Max of 2 points for a solution using explicit matrix multiplication.

\textit{Hint: Using parts a and b, note that CSIGN is the same when applied in either direction.}
  \end{question}
  \tcblower{}
  \begin{proof}
    Note that the CSIGN gate has a non-trivial effect (out of all basis states) only when both the control and target qubits are in the state $\ket{1}$. This is symmetric in both the control and target qubits and the effect is just going from $\ket{11} \to -\ket{11}$, which is also symmetric in the control and target. Therefore, the CSIGN gate is equivalent even if we swap the control and target qubits.\par
    Now we show the equality using the following circuit transformations:

    \begin{minipage}[b]{0.53\textwidth}
      \centering
      \begin{quantikz}
        \qw{} & \gate{H} & \ctrl{1} & \gate{H} & \qw{}\\
        \qw{} & \gate{H} & \targ{} & \gate{H} & \qw{}
      \end{quantikz}
    \end{minipage}
    $\equiv$
    \begin{minipage}[b]{0.47\textwidth}
      \begin{quantikz}
        \qw{} & \gate{H} & \gategroup[2,steps=3,style={fill=blue!20},background]{CSIGN} & \ctrl{1} & & \gate{H} & \qw{}\\
        \qw{} & & \gate{H} & \targ{} & \gate{H} & & \qw{}
      \end{quantikz}
    \end{minipage}
    $\equiv$
    \begin{minipage}[b]{0.5\textwidth}
      \centering
      \begin{quantikz}
        \qw{} & \gate{H} & \gate{H}\gategroup[2,steps=3,style={fill=blue!20},background]{CSIGN} & \targ{} & \gate{H} & \gate{H} & \qw{}\\
        \qw{} & & & \ctrl{-1} & & & \qw{}
      \end{quantikz}
    \end{minipage}
    $\equiv$
    \begin{minipage}[b]{0.5\textwidth}
      \begin{quantikz}
        \ghost{H} & \targ{} & \qw{}\\
        \ghost{H} & \ctrl{-1} & \qw{}
      \end{quantikz}
    \end{minipage}

    Hence, proved.
  \end{proof}
\end{solution}
