\mathchardef\mhyphen="2D
\documentclass[11pt]{article}
\usepackage[english]{babel}
\usepackage{minted}
\usepackage{amsfonts}
\usepackage{amsmath}
\usepackage{amsthm}
\usepackage{amssymb}
\usepackage{graphicx}
\usepackage{subcaption}
\usepackage[hypcap=false]{caption}
\usepackage{booktabs}
\usepackage[left=25mm, top=25mm, bottom=25mm, right=25mm]{geometry}
\usepackage{soul}
\usepackage{algorithm}
\usepackage{algpseudocode}
\usepackage[most]{tcolorbox}
\usepackage[colorlinks=true,linkcolor=darkcyan,filecolor=darkcerulean,urlcolor=magenta]{hyperref}
\usepackage{braket}
\usepackage{quantikz}
\usepackage{enumitem}
\usepackage{svg}

\newcommand{\calA}{\mathcal{A}}
\newcommand{\calB}{\mathcal{B}}
\newcommand{\calC}{\mathcal{C}}
\newcommand{\calD}{\mathcal{D}}
\newcommand{\calE}{\mathcal{E}}
\newcommand{\calF}{\mathcal{F}}
\newcommand{\calG}{\mathcal{G}}
\newcommand{\calH}{\mathcal{H}}
\newcommand{\calI}{\mathcal{I}}
\newcommand{\calJ}{\mathcal{J}}
\newcommand{\calK}{\mathcal{K}}
\newcommand{\calL}{\mathcal{L}}
\newcommand{\calM}{\mathcal{M}}
\newcommand{\calN}{\mathcal{N}}
\newcommand{\calO}{\mathcal{O}}
\newcommand{\calP}{\mathcal{P}}
\newcommand{\calQ}{\mathcal{Q}}
\newcommand{\calR}{\mathcal{R}}
\newcommand{\calS}{\mathcal{S}}
\newcommand{\calT}{\mathcal{T}}
\newcommand{\calU}{\mathcal{U}}
\newcommand{\calV}{\mathcal{V}}
\newcommand{\calW}{\mathcal{W}}
\newcommand{\calX}{\mathcal{X}}
\newcommand{\calY}{\mathcal{Y}}
\newcommand{\calZ}{\mathcal{Z}}

\newcommand{\bfA}{\mathbf{A}}
\newcommand{\bfB}{\mathbf{B}}
\newcommand{\bfC}{\mathbf{C}}
\newcommand{\bfD}{\mathbf{D}}
\newcommand{\bfE}{\mathbf{E}}
\newcommand{\bfH}{\mathbf{H}}
\newcommand{\bfI}{\mathbf{I}}
\newcommand{\bfS}{\mathbf{S}}
\newcommand{\bfP}{\mathbf{P}}
\newcommand{\bfQ}{\mathbf{Q}}
\newcommand{\bfU}{\mathbf{U}}
\newcommand{\bfX}{\mathbf{X}}
\newcommand{\bfY}{\mathbf{Y}}
\newcommand{\bfZ}{\mathbf{Z}}
\newcommand{\bfv}{\mathbf{v}}
\newcommand{\bfu}{\mathbf{u}}
\newcommand{\bfdelta}{\mathbf{\Delta}}
\newcommand{\bfpi}{\mathbf{\Pi}}


\newcommand{\N}{\mathbb{N}}
\newcommand{\z}{\mathbb{Z}}
\newcommand{\I}{\mathbb{I}}
\newcommand{\C}{\mathbb{C}}

\newcommand{\keygen}{\mathsf{KeyGen}}
\newcommand{\enc}{\mathsf{Enc}}
\newcommand{\dec}{\mathsf{Dec}}
\newcommand{\negl}{\mathsf{negl}}
\newcommand{\commit}{\mathsf{Commit}}
\newcommand{\ccommit}{\mathsf{C\text{-}Commit}}

\newcommand{\setup}{\mathsf{Setup}}
\newcommand{\lsetup}{\mathsf{Setup\text{-}Lossy}}

\newcommand{\samplemat}{\mathsf{Sample\text{-}Dirac\text{-}Matrix}}
\newcommand{\eval}{\mathsf{Eval}}
\newcommand{\obf}{\mathsf{Obf}}

\newcommand{\phybb}[1]{p_{\mathrm{hyb}, #1}}
\newcommand{\lin}{\ell_{\mathrm{in}}}
\newcommand{\lout}{\ell_{\mathrm{out}}}
\newcommand{\bit}{\{0,1\}}
\newcommand{\sd}{\mathsf{SD}}
\newcommand{\lwe}{\mathsf{LWE}_{n,m,q,\chi}}
\newcommand{\sslwe}[4]{\mathsf{ss\text{-}LWE}_{#1,#2,#3,#4}}
\newcommand{\sslwec}{\sslwe{n}{m}{q}{\chi}}

\newcommand{\unif}[1]{\mathsf{Unif}_{\left[-#1, #1\right]}}
\newcommand{\func}[2]{\mathsf{Func}[#1, #2]}
\newcommand{\perm}[2]{\mathsf{Perm}[#1, #2]}
\newcommand{\ct}{\mathsf{ct}}
\newcommand{\Finverse}{F^{-1}}

\newcommand{\nqss}{\mathsf{No\mhyphen Query \mhyphen Semantic \mhyphen Security}}

\newcommand{\Tr}{\mathrm{Tr}}
\newcommand{\trace}[1]{\Tr\left( #1 \right)}

\newcommand{\prob}[1]{\Pr\left[ #1 \right]}

\newcommand{\brakett}[2]{\braket{#1|#2}}
\newcommand{\ketbraa}[2]{\ket{#1}\bra{#2}}
\newcommand{\ketbra}[1]{\ketbraa{#1}{#1}}

\newcommand{\lddh}{\mathcal{L}_{DDH}}
\newcommand{\lnddh}{\mathcal{L}_{nDDH}}
\newcommand{\lddhkt}{\mathcal{L}_{DDH,k,t}}

\newlength{\protowidth}
\newcommand{\pprotocol}[4]{
{\begin{center}
\setlength{\protowidth}{\textwidth}
\addtolength{\protowidth}{-3\intextsep}

\fbox{
        \small
        \hbox{\quad
        \begin{minipage}{\protowidth}
    \begin{center}
    {\bf #1}
    \end{center}
        #4
        \end{minipage}
        \quad}
        }
        \captionof{figure}{\label{#3} #2}
\end{center}
} }

\newcommand{\defbox}[1]{
{\begin{center}
\setlength{\protowidth}{\textwidth}
\addtolength{\protowidth}{-3\intextsep}

\fcolorbox{darkcerulean}{cottoncandy}{
        \small
        \hbox{\quad
        \begin{minipage}{\protowidth}
    
        #1
        \end{minipage}
        \quad}
        }
\end{center}
        } }

\newcommand{\protocol}[4]{
\pprotocol{#1}{#2}{#3}{#4} }

\newtheorem{theorem}{Theorem}[section]
\newtheorem{claim}[theorem]{Claim}
\newtheorem{fact}[theorem]{Fact}
\newtheorem{definition}[theorem]{Definition}
\newtheorem*{question}{Question}

\newtcolorbox[auto counter,number within=section]{solution}[1][]{every float=\centering,breakable,enhanced,adjusted title={Question \thetcbcounter},#1,colback=codegray,colframe=darkcerulean!50!black}

\linespread{1.0}

\definecolor{codegray}{rgb}{0.98,0.97,0.93}
\definecolor{cottoncandy}{rgb}{1.0, 0.84, 0.95}
\definecolor{darkcerulean}{rgb}{0.03, 0.27, 0.49}
\definecolor{darkcyan}{rgb}{0.0, 0.50, 0.45}

\title{C S 358H: Intro to Quantum Information Science}
\author{Sayam Sethi}
\date{November 2024}

\begin{document}

\maketitle

\tableofcontents

\pagenumbering{arabic}

\newpage

\section{Generalized Deustch-Jozsa Problem}
Recall that in the Deustch-Jozsa problem we are given oracle access to an unknown
function $f \colon \{0,1\} \to \{0,1\}$ and wish to determine if it is constant or
balanced. In the generalized Deustch-Jozsa problem we are given oracle access to a
function $f\colon \{0,1\}^n \to \{0,1\}$, a function that maps $n$-bit strings to a
single bit. We're promised that the function is either constant or balanced, where
balanced means that it has an equal number of 0 and 1 outputs ($2^{n-1}$ of each).
\begin{solution}{Part a}\label{ques:1a}
  \begin{question}
    Of the following matrices, which ones are stochastic? Which
    ones are unitary?

    \begin{gather*}
    A =
    \begin{bmatrix}
    1 & 0\\
    0 & 0
    \end{bmatrix}\!,\;
    B=
    \begin{bmatrix}
    0 & 1\\
    1 & 0
    \end{bmatrix}\!,\;
    C=
    \begin{bmatrix}
    1 & \frac{2}{3}\\[0.1em]
    0 & \frac{1}{3}
    \end{bmatrix}\!,\;
    D=
    \begin{bmatrix}
    1 & 0\\
    0 & i
    \end{bmatrix}\!,\\
    E =
    \begin{bmatrix}
    2 & \frac{1}{2}\\[0.1em]
    -1 & \frac{1}{2}
    \end{bmatrix}\!,\;
    F = \frac{1}{\sqrt{2}}
    \begin{bmatrix}
    1 & 1\\
    1 & -1
    \end{bmatrix}\!,\;
    G=
    \begin{bmatrix}
    \frac{3}{5} & \frac{4}{5}\\[0.1em]
    \frac{4}{5} & -\frac{3}{5}
    \end{bmatrix}\!,\;
    H=
    \begin{bmatrix}
    \frac{3i}{5} & \frac{4}{5}\\[0.1em]
    \frac{4}{5} & -\frac{3i}{5}
    \end{bmatrix}
    \end{gather*}
  \end{question}
  \tcblower{}
  \begin{proof}[Solution]
    A matrix $\mathbf{A} = (a_{ij})$ is stochastic iff $\sum_{i} a_{ij} = 1 \wedge a_{ij} \geq 0$. Therefore, the stochastic matrices are $B, C$.\par
    A matrix $\mathbf{A}$ is unitary iff $\mathbf{A}^\dag\mathbf{A} = \mathbf{I}$. Therefore, the unitary matrices are $B, D, F, G$.\par
    Note that matrix $A, H$ is neither stochastic nor unitary.
  \end{proof}
\end{solution}

\begin{solution}{Part b}\label{ques:1b}
  \begin{question}
    Show that any stochastic matrix that is also unitary must be a permutation matrix.
  \end{question}
  \tcblower{}
  \begin{proof}
    Let $\mathbf{A}$ be a matrix that is stochastic and unitary. This implies,
    \begin{equation}
      \begin{split}
        \mathbf{A}^\dag\mathbf{A} &= \mathbf{I}\\
        \sum_{i} a_{ij} = 1 &\wedge a_{ij} \geq 0
      \end{split}
    \end{equation}
    Representing the above properties in terms of the matrix elements $\mathbf{A} = (a_{ij})$, we get the following,
    \begin{equation}
      \forall\ i: \sum_{i} a_{ij}\cdot a_{ji} = 1
      \label{eq:11}
    \end{equation}
    \begin{equation}
      \forall\ i \neq k: \sum_{i} a_{ij}\cdot a_{ki} = 0
      \label{eq:12}
    \end{equation}
    \begin{equation}
      \forall\ i: \exists\ p_i: a_{ip_i} \neq 0
      \label{eq:13}
    \end{equation}
    Now, from \eqref{eq:12} and \eqref{eq:13}, we get that $a_{ij} = 0\ \forall\ i \neq p_j$, which implies that $a_{ip_i} = 1$ from \eqref{eq:11}. Therefore, $\mathbf{A}$ is a permutation matrix with $\Pi = \{p_i\}$.
  \end{proof}
\end{solution}

\begin{solution}{Part c}\label{ques:1c}
  \begin{question}
    Stochastic matrices preserve the $1$-norms of nonnegative vectors, while
unitary matrices preserve $2$-norms. Give an example of a $2\times
2$\ matrix, other than the identity matrix, that preserves the $4$-norm of
real vectors $\begin{bsmallmatrix} a\\b\end{bsmallmatrix}$: that is, $a^{4}+b^{4}$.
  \end{question}
  \tcblower{}
  \begin{proof}[Solution]
    $\mathbf{A} = \begin{pmatrix}
      0 & 1\\
      1 & 0
      \end{pmatrix}$ preserves the $4$-norm of the vector $\begin{psmallmatrix} a\\b\end{psmallmatrix}$.
  \end{proof}
\end{solution}

  \begin{solution}{Part d}\label{ques:1d}
  \begin{question}
    Give a characterization of all real matrices that preserve the
$4$\textit{-norms} of real vectors. \ Hopefully, your characterization will
help explain why preserving the $2$-norm, as quantum mechanics does, leads to
a much richer set of transformations than preserving the $4$-norm does.
  \end{question}
  \tcblower{}
  \begin{proof}
  \end{proof}
\end{solution}


\newpage

\section{The Birthday Paradox}
 Your favorite local radio station is running a new give-away contest that works as follows: Each day at 5pm the phone lines open up to the public to call in and leave their name and birth-date which is then added to a running list. The contest runs until a matching birth-date (month and date) between any two contestants on the list is found. At that point the contest closes and everyone on the list up to that point is a winner. Assume that every birthday is equally likely and that no contestants are born during a leap year (so that we can ignore Feb. 29\textsuperscript{th}).  

As usual, you must show or explain your work for each part. You are free to use numerical software of your choice to help solve the problem
\begin{solution}{Part a}\label{ques:2a}
  \begin{question}
    Measure in the $\{\ket{0}, \ket{1}\}$-basis:
    \begin{minipage}[t]{\textwidth}
      \begin{quantikz}
        \lstick{$\ket{0}$} & \gate{H} & \gate{Z} & \gate{H} & \gate{P} & \meter{}
      \end{quantikz}
    \end{minipage}
  \end{question}
  \tcblower{}
  \begin{proof}[Solution]
    The state of the qubit right before the measurement will be, $\ket{\psi} = i\ket{1}$ which on measurement in the standard basis will give the output $1$.
  \end{proof}
\end{solution}

\begin{solution}{Part b}\label{ques:2b}
  \begin{question}
    Measure in the $\{\ket{+}, \ket{-}\}$-basis:
    \begin{minipage}[t]{\textwidth}
      \begin{quantikz}
        \lstick{$\ket{0}$} & \gate{R_{\pi/4}} & \gate{Z} & \gate{Y} & \gate{H} & \meter{}
      \end{quantikz}
    \end{minipage}
  \end{question}
  \tcblower{}
  \begin{proof}[Solution]
    The state of the qubit right before the measurement will be, $\ket{\psi} = i\ket{0}$ which on measurement in the Hadamard basis will give either $\ket{+}$ or $\ket{-}$ with equal probability.
  \end{proof}
\end{solution}

\begin{solution}{Part c}\label{ques:2c}
  \begin{question}
    Measure in the $\{\ket{+}, \ket{-}\}$-basis:
    \begin{minipage}[t]{\textwidth}
      \begin{quantikz}
        \lstick{$\ket{+}$} & \gate{T} & \gate{H} & \meter{}
      \end{quantikz}
    \end{minipage}
  \end{question}
  \tcblower{}
  \begin{proof}[Solution]
    Since we have to measure in the $\ket{+}, \ket{-}$ basis, we can add a Hadamard gate before the measurement and measure in the standard basis. This will cancel out the $H$ gate and the resultant circuit is just executing a $T$ gate on the $\ket{+}$ state. The resultant state after applying the gate is $\frac{1}{\sqrt{2}}(\ket{0} + e^{i\pi/4}\ket{1})$. Therefore, the probabilites of getting $\ket{+}$ and $\ket{-}$ is $50\%$ each.\par
    Since the question also asks for the state before measurement, the state before measurement will be $\frac{1}{\sqrt{2}}(\ket{+} + e^{i\pi/4}\ket{-})$.
  \end{proof}
\end{solution}

\begin{solution}{Part d}\label{ques:2d}
  \begin{question}
    Measure in the $\{\ket{i}, \ket{-i}\}$-basis:
    \begin{minipage}[t]{\textwidth}
      \begin{quantikz}
        \lstick{$\ket{+}$} & \gate{T} & \gate{Z} & \gate{T} & \meter{}
      \end{quantikz}
    \end{minipage}
  \end{question}
  \tcblower{}
  \begin{proof}[Solution]
    The $T$ gate is a rotation along the $Z$ axis by $\pi/8$ radians and the $Z$ gate is a rotation along the Z axis by $\pi/2$ radians. Therefore, the entire circuit can be seen as a rotation along the $Z$ axis by $3\pi/4$ radians. Therefore, the resultant state is $\frac{1}{\sqrt{2}}(\ket{0} + e^{i3\pi/2}\ket{1}) = \frac{1}{\sqrt{2}}(\ket{0} - i\ket{1}) = \ket{-i}$. Therefore the probability of obtaining $\ket{-i}$ is $1$.
  \end{proof}
\end{solution}


\newpage

\section{Two Secrets}
\begin{solution}{Part a}\label{ques:3a}
  \begin{question}
    Normalize the state $\ket{0} + \ket{+}$.
  \end{question}
  \tcblower{}
  \begin{proof}[Solution]
    The state is equal to $(1 + \frac{1}{\sqrt{2}})\ket{0} + \frac{1}{\sqrt{2}}\ket{1}$. Therefore, the normalization factor will be $\sqrt{2 + \sqrt{2}}$ and the normalized state will be $\frac{1}{\sqrt{2\sqrt{2}}}\left(\sqrt{\sqrt{2} + 1}\ket{0} + \sqrt{\sqrt{2} - 1}\ket{1}\right)$.
  \end{proof}
\end{solution}

\begin{solution}{Part b}\label{ques:3b}
  \begin{question}
    We say a quantum state vector $\ket{\psi}$ is an eigenvector or eigenstate of a matrix $\Lambda$ if the following equation holds for some number $\lambda$:
\[
\Lambda \ket{\psi} = \lambda \ket{\psi}
\]

$\lambda$ is called the eigenvalue of $\ket{\psi}$. Show that the normalized form of the state from part a) is an eigenstate of the $H$ gate. What is the eigenvalue?
  \end{question}
  \tcblower{}
  \begin{proof}
    If we represent the state in part a as $\ket{\psi} = a(\ket{0} + \ket{+})$, where $a = \frac{1}{\sqrt{2 + \sqrt{2}}}$, then we get,
    \begin{equation}
      \begin{split}
        H\ket{\psi} &= H\left(a(\ket{0} + \ket{+})\right)\\
        &= a(H(\ket{0} + \ket{+}))\\
        &= a(\ket{+} + \ket{0}) = 1\cdot\ket{\psi}
      \end{split}
    \end{equation}
    Therefore, $\ket{\psi}$ is an eigenstate of $H$ with an eigenvalue of $1$.
  \end{proof}
\end{solution}

\begin{solution}{Part c}\label{ques:3c}
  \begin{question}
    What single-qubit states are reachable from $\ket{0}$ using only $H$ and $P$, i.e. via any sequence of $H$ and $P$ gates? Are there finitely or infinitely many? Characterize all of the reachable states, up to global phase.
\\ Hint: Start by just playing with applying different sequences of matrices to the $\ket{0}$ state and look for a pattern.
  \end{question}
  \tcblower{}
  \begin{proof}
    There are a finitely many states that can be reached from $\ket{0}$ using only $H$ and $P$. We first show the following,
    \begin{equation}
      PH\ket{0} = P\ket{+} = \frac{1}{\sqrt{2}}H(\ket{0} + i\ket{1}) = \ket{i}
      \label{eq:ph}
    \end{equation}
    Similarly, we have,
    \begin{equation}
      P^2H\ket{0} = P^2\ket{+} = \ket{-}
      \label{eq:p2h}
    \end{equation}
    Also,
    \begin{equation}
      P^3H\ket{0} = P^3\ket{+} = \ket{-i}
      \label{eq:p3h}
    \end{equation}
    We also state the following from Equations~\ref{eq:ph},~\ref{eq:p2h},~\ref{eq:p3h},
    \begin{equation}
      \begin{split}
        HPH\ket{0} &= \ket{-i}\\
        HP^2H\ket{0} &= \ket{1}\\
        HP^3H\ket{0} &= \ket{i}
      \end{split}
      \label{eq:hpih}
    \end{equation}

    Also note that $P^4 = I$ and $H^2 = I$. Therefore, a non-trivial application of a unitary made up of a sequence of $H$ and $P$ gates will be of the form $H^{h}\cdot P^{p_n}\cdot H\cdot P^{p_{n-1}}\cdot H\cdot P^{p_{n-2}}\cdots H\cdot P^{p_1}\cdot H$ where $h\in \{0, 1\}\ \wedge\ \forall i:\ p_i \in \{1, 2, 3\}$. In words, we start with a Hadamard gate and we alternate between phase gates and Hadamards in between and may or may not end with a Hadamard gate. Note that starting with a phase gate has no effect on the $\ket{0}$ state and hence we can ignore its effect. Therefore, the only set of states that are reachable by such a unitary belong to the set $\mathbb{S} = \{\ket{0}, \ket{1}, \ket{+}, \ket{-}, \ket{i}, \ket{-i}\}$. We can easily prove this by induction on $i$.
  \end{proof}
\end{solution}


\newpage

\section{Bernstein-Vazirani}
In the Bernstein-Vazirani problem, recall that we're
given oracle access to a Boolean function $f\colon\{0,1\}^n\to\{0,1\}$. We're
promised that there exists a ``secret string'' $s \in \{0,1\}^n$ such that
$f(x)=s \cdot x \bmod{2}$ for all $x$. The problem is to recover $s$. The
Bernstein-Vazirani algorithm solves this problem with just a single quantum query
to $f$. Consider a variant of this problem where we are promised that $f(x) = s\cdot x \bmod{2}$ for \emph{at most} a $(1 - \epsilon)$ fraction of the inputs $x$,
and that $f(x) = (s\cdot x + 1) \bmod{2}$ for the remaining $\epsilon$ fraction of
inputs.
\begin{solution}{Part a}\label{ques:4a}
  \begin{question}
    But what if $\ket{\psi}$ is either $\ket{0}$ or $\ket{+}$ (with equal probability)? Give the protocol that distinguishes the two states with with a failure probability of $\sin^2(\frac{\pi}{8})\approx .146$. Show explicitly that your protocol achieves this failure probability.
\\ Note that when we ask for a protocol, we mean some step-by-step algorithm that ends by outputting ``I think this was $\ket{0}$'' or ``I think this was $\ket{+}$''.
\\ Hint: Read Section 5.2 of the textbook.
  \end{question}
  \tcblower{}
  \begin{proof}
    We propose the following protocol,
    \protocol{Protocol to distinguish between $\ket{0}$ and $\ket{+}$}{Distinguishing protocol}{proto:diff}{
      \begin{enumerate}
        \item Apply the $R_X(\pi/8)$ gate to the input state $\ket{\psi}$ (i.e., rotate the state by $\pi/8$ along the $X$ axis).
        \item Measure the state in the standard basis.
        \item If the measurement result is $\ket{0}$, output $\ket{0}$, else output $\ket{+}$.
      \end{enumerate}
    }
    We now prove that the failure probability of this protocol is $\sin^2(\pi/8)$. If we had the $\ket{0}$ state, then the state of the qubit after applying the rotation gate is $\cos\frac{\pi}{8}\ket{0} + \sin\frac{\pi}{8}\ket{1}$. Alternatively, if we started with the $\ket{+}$ state, the state would be $\cos\frac{3\pi}{8}\ket{0} + \sin\frac{3\pi}{8}\ket{1}$. Now, the failure probability can be computed as,
    \begin{equation}
      \begin{split}
        \prob{\text{failure}} &= \frac{1}{2}\cdot\prob{\text{failure }| \ket{\psi} = \ket{0}} + \frac{1}{2}\cdot\prob{\text{failure }| \ket{\psi} = \ket{+}}\\
        &= \frac{1}{2}\cdot\sin^2\frac{\pi}{8} + \frac{1}{2}\cdot\cos^2\frac{3\pi}{8}\\
        &= \frac{1}{2}\cdot\sin^2\frac{\pi}{8} + \frac{1}{2}\cdot\sin^2\frac{\pi}{8} = \sin^2\frac{\pi}{8}
      \end{split}
    \end{equation}
    Hence, we have shown that the failure probability of this protocol is $\sin^2(\pi/8)$.
  \end{proof}
\end{solution}

\begin{solution}{Part b}\label{ques:4b}
  \begin{question}
    Prove that this is optimal.
  \end{question}
  \tcblower{}
  \begin{proof}
    Any protocol that will be used to distinguish the two states can be represented as a unitary on some $n$ qubits, followed by measurements in the standard basis at the end. Any intermediate measurements can be deferred to the end by adding more qubits in the circuit (for each intermediate measurement, add an extra qubit, perform a CNOT between the qubit to be measured and the new qubit and instead measure the new qubit; now this new qubit isn't involved in any further operations so its measurement can be done at any point, we do it at the end along with all other measurements).\par
    Therefore, any protocol has the effect of performing a measurement in a certain orthonormal basis. However, we know that no unitary can change the inner product between two states. This implies that the angle between the two states is still fixed at $\theta = \cos^{-1}\left|\brakett{\psi_1}{\psi_2}\right|$. Thus, the most optimal basis to measure the two states will have each orthonormal state at an angle of some $\alpha$ and $\alpha + \theta$ from the two states (assuming that $\theta$ is acute, the argument is similar for an obtuse $\theta$). Therefore the least failure probability for any protocol is $2\cdot \frac{1}{2}\cdot \cos^2\alpha = \cos^2(\pi/2 - \theta/2) = \sin^2\theta/2$.\par
    For the states $\ket{0}$ and $\ket{+}$, the angle is $\pi/4$ and therefore the best failure probability is $\sin^2\pi/8$.
  \end{proof}
\end{solution}

\begin{solution}{Part c}\label{ques:4c}
  \begin{question}
    What is the failure probability if you measure in the $\{\ket{0}, \ket{1}\}$ basis?
  \end{question}
  \tcblower{}
  \begin{proof}[Solution]
    If we measure in the $\{\ket{0}, \ket{1}\}$ basis and guess that the state was $\ket{+}$ only if the measurement result is $\ket{1}$, the failure probability can be computed as,
    \begin{equation}
      \begin{split}
        \prob{\text{failure}} &= \frac{1}{2}\cdot\prob{\text{failure }| \ket{\psi} = \ket{0}} + \frac{1}{2}\cdot\prob{\text{failure }| \ket{\psi} = \ket{+}}\\
        &= 0 + \frac{1}{2}\cdot\frac{1}{2}\text{, since there is a 50\% probability of getting }\ket{1}\\
        &= \frac{1}{4}
      \end{split}
    \end{equation}
  \end{proof}
\end{solution}


\end{document}
