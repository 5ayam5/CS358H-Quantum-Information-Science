\begin{solution}{Part a}\label{ques:1a}
  \begin{question}
    Of the following matrices, which ones are stochastic? Which
    ones are unitary?

    \begin{gather*}
    A =
    \begin{bmatrix}
    1 & 0\\
    0 & 0
    \end{bmatrix}\!,\;
    B=
    \begin{bmatrix}
    0 & 1\\
    1 & 0
    \end{bmatrix}\!,\;
    C=
    \begin{bmatrix}
    1 & \frac{2}{3}\\[0.1em]
    0 & \frac{1}{3}
    \end{bmatrix}\!,\;
    D=
    \begin{bmatrix}
    1 & 0\\
    0 & i
    \end{bmatrix}\!,\\
    E =
    \begin{bmatrix}
    2 & \frac{1}{2}\\[0.1em]
    -1 & \frac{1}{2}
    \end{bmatrix}\!,\;
    F = \frac{1}{\sqrt{2}}
    \begin{bmatrix}
    1 & 1\\
    1 & -1
    \end{bmatrix}\!,\;
    G=
    \begin{bmatrix}
    \frac{3}{5} & \frac{4}{5}\\[0.1em]
    \frac{4}{5} & -\frac{3}{5}
    \end{bmatrix}\!,\;
    H=
    \begin{bmatrix}
    \frac{3i}{5} & \frac{4}{5}\\[0.1em]
    \frac{4}{5} & -\frac{3i}{5}
    \end{bmatrix}
    \end{gather*}
  \end{question}
  \tcblower{}
  \begin{proof}[Solution]
    A matrix $\mathbf{A} = (a_{ij})$ is stochastic iff $\sum_{i} a_{ij} = 1 \wedge a_{ij} \geq 0$. Therefore, the stochastic matrices are $B, C$.\par
    A matrix $\mathbf{A}$ is unitary iff $\mathbf{A}^\dag\mathbf{A} = \mathbf{I}$. Therefore, the unitary matrices are $B, D, F, G$.\par
    Note that matrix $A, H$ is neither stochastic nor unitary.
  \end{proof}
\end{solution}

\begin{solution}{Part b}\label{ques:1b}
  \begin{question}
    Show that any stochastic matrix that is also unitary must be a permutation matrix.
  \end{question}
  \tcblower{}
  \begin{proof}
    Let $\mathbf{A}$ be a matrix that is stochastic and unitary. This implies,
    \begin{equation}
      \begin{split}
        \mathbf{A}^\dag\mathbf{A} &= \mathbf{I}\\
        \sum_{i} a_{ij} = 1 &\wedge a_{ij} \geq 0
      \end{split}
    \end{equation}
    Representing the above properties in terms of the matrix elements $\mathbf{A} = (a_{ij})$, we get the following,
    \begin{equation}
      \forall\ i: \sum_{i} a_{ij}\cdot a_{ji} = 1
      \label{eq:11}
    \end{equation}
    \begin{equation}
      \forall\ i \neq k: \sum_{i} a_{ij}\cdot a_{ki} = 0
      \label{eq:12}
    \end{equation}
    \begin{equation}
      \forall\ i: \exists\ p_i: a_{ip_i} \neq 0
      \label{eq:13}
    \end{equation}
    Now, from \eqref{eq:12} and \eqref{eq:13}, we get that $a_{ij} = 0\ \forall\ i \neq p_j$, which implies that $a_{ip_i} = 1$ from \eqref{eq:11}. Therefore, $\mathbf{A}$ is a permutation matrix with $\Pi = \{p_i\}$.
  \end{proof}
\end{solution}

\begin{solution}{Part c}\label{ques:1c}
  \begin{question}
    Stochastic matrices preserve the $1$-norms of nonnegative vectors, while
unitary matrices preserve $2$-norms. Give an example of a $2\times
2$\ matrix, other than the identity matrix, that preserves the $4$-norm of
real vectors $\begin{bsmallmatrix} a\\b\end{bsmallmatrix}$: that is, $a^{4}+b^{4}$.
  \end{question}
  \tcblower{}
  \begin{proof}[Solution]
    $\mathbf{A} = \begin{pmatrix}
      0 & 1\\
      1 & 0
      \end{pmatrix}$ preserves the $4$-norm of the vector $\begin{psmallmatrix} a\\b\end{psmallmatrix}$.
  \end{proof}
\end{solution}

  \begin{solution}{Part d}\label{ques:1d}
  \begin{question}
    Give a characterization of all real matrices that preserve the
$4$\textit{-norms} of real vectors. \ Hopefully, your characterization will
help explain why preserving the $2$-norm, as quantum mechanics does, leads to
a much richer set of transformations than preserving the $4$-norm does.
  \end{question}
  \tcblower{}
  \begin{proof}
  \end{proof}
\end{solution}
