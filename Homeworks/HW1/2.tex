\begin{solution}{Part a}\label{ques:2a}
  \begin{question}
    Calculate the tensor product 
    \[
    \begin{bmatrix}
    \frac{2}{3}\\[0.1em]
    \frac{1}{3}
    \end{bmatrix}\otimes
    \begin{bmatrix}
    \frac{1}{5}\\[0.1em]
    \frac{4}{5}
    \end{bmatrix}.
    \]
  \end{question}
  \tcblower{}
  \begin{proof}[Solution]
    \begin{equation}
      \begin{pmatrix}
        \frac{2}{3}\\[0.1em]
        \frac{1}{3}
      \end{pmatrix}
      \otimes
      \begin{pmatrix}
        \frac{1}{5}\\[0.1em]
        \frac{4}{5}
      \end{pmatrix}
      =
      \begin{pmatrix}
        \frac{2}{3}\begin{pmatrix}
          \frac{1}{5}\\[0.1em]
          \frac{4}{5}
        \end{pmatrix}\\[1em]
        \frac{1}{3}\begin{pmatrix}
          \frac{1}{5}\\[0.1em]
          \frac{4}{5}
        \end{pmatrix}
      \end{pmatrix}
      =
      \begin{pmatrix}
        \frac{2}{15}\\[0.1em]
        \frac{8}{15}\\[0.1em]
        \frac{1}{15}\\[0.1em]
        \frac{4}{15}
      \end{pmatrix}
    \end{equation}
  \end{proof}
\end{solution}

\begin{solution}{Part b}\label{ques:2b}
  \begin{question}
    Of the following length-$4$ vectors, decide which ones are factorizable as a tensor product of two $2 \times 1$ vectors, and factorize them. \ (Here the vector entries
    should be thought of as labeled by $00$, $01$, $10$, and $11$ respectively.)%
    \[
    A=
    \begin{bmatrix}
      \frac{2}{9}\\[0.1em]
      \frac{1}{9}\\[0.1em]
      \frac{4}{9}\\[0.1em]
      \frac{2}{9}
    \end{bmatrix}\!,\;
    B=
    \begin{bmatrix}
      0\\
      1\\
      0\\
      0
    \end{bmatrix}\!,\;
    C=
    \begin{bmatrix}
      \frac{1}{4}\\[0.1em]
      \frac{1}{4}\\[0.1em]
      \frac{1}{4}\\[0.1em]
      \frac{1}{4}
    \end{bmatrix}\!,\;
    D=
    \begin{bmatrix}
      0\\
      \frac{1}{2}\\[0.1em]
      \frac{1}{2}\\
      0
    \end{bmatrix}\!,\;
    E=
    \begin{bmatrix}
      0\\
      \frac{1}{2}\\
      0\\
      \frac{1}{2}
    \end{bmatrix}.
    \]
  \end{question}
  \tcblower{}
  \begin{proof}[Solution]
    The following matrices can be factorized as a tensor product of two $2 \times 1$ vectors:
    \begin{equation}
      A = \begin{pmatrix}
            \frac{1}{3}\\[0.1em]
            \frac{2}{3}
          \end{pmatrix}
          \otimes
          \begin{pmatrix}
            \frac{2}{3}\\[0.1em]
            \frac{1}{3}
          \end{pmatrix}
    \end{equation}
    \begin{equation}
      B = \begin{pmatrix}
            1\\
            0
          \end{pmatrix}
          \otimes
          \begin{pmatrix}
            0\\
            1
          \end{pmatrix}
    \end{equation}
    \begin{equation}
      C = \begin{pmatrix}
            \frac{1}{2}\\[0.1em]
            \frac{1}{2}
          \end{pmatrix}
          \otimes
          \begin{pmatrix}
            \frac{1}{2}\\[0.1em]
            \frac{1}{2}
          \end{pmatrix}
    \end{equation}
    \begin{equation}
      E = \begin{pmatrix}
            \frac{1}{2}\\[0.1em]
            \frac{1}{2}
          \end{pmatrix}
          \otimes
          \begin{pmatrix}
            0\\
            1
          \end{pmatrix}
    \end{equation}
    Matrix $D$ cannot be factorized as a tensor product of two $2 \times 1$ vectors.
  \end{proof}
\end{solution}

\begin{solution}{Part c}\label{ques:2c}
  \begin{question}
    Prove that there's no $2 \times 2$ \emph{real} matrix $A$ such
    that
    \[
    A^2=
    \begin{bmatrix}
      1 & 0\\
      0 & -1
    \end{bmatrix}\!.
    \]
    This observation perhaps helps to explain why the complex numbers play such a
    central role in quantum mechanics.
  \end{question}
  \tcblower{}
  \begin{proof}
    We will prove this by contradiction. Let there be a real matrix $\mathbf{A}$ such that $\mathbf{A}^2 = \begin{bmatrix} 1 & 0\\ 0 & -1 \end{bmatrix}$. Notice that $det(\mathbf{A}^2) = -1$. However, we know that $det(\mathbf{A}^2) = det(\mathbf{A})^2$. Therefore, $det(\mathbf{A})^2 = -1 \implies det(\mathbf{A}) = \pm i$, which is not possible since $\mathbf{A}$ is a real matrix and hence its determinant will be real. This is a contradiction and hence there is no real matrix satisfying the condition.
  \end{proof}
\end{solution}
