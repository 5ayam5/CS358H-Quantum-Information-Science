\begin{solution}{Part a}\label{ques:3a}
  \begin{question}
    Normalize the state $\ket{0} + \ket{+}$.
  \end{question}
  \tcblower{}
  \begin{proof}[Solution]
    The state is equal to $(1 + \frac{1}{\sqrt{2}})\ket{0} + \frac{1}{\sqrt{2}}\ket{1}$. Therefore, the normalization factor will be $\sqrt{2 + \sqrt{2}}$ and the normalized state will be $\frac{1}{\sqrt{2\sqrt{2}}}\left(\sqrt{\sqrt{2} + 1}\ket{0} + \sqrt{\sqrt{2} - 1}\ket{1}\right)$.
  \end{proof}
\end{solution}

\begin{solution}{Part b}\label{ques:3b}
  \begin{question}
    We say a quantum state vector $\ket{\psi}$ is an eigenvector or eigenstate of a matrix $\Lambda$ if the following equation holds for some number $\lambda$:
\[
\Lambda \ket{\psi} = \lambda \ket{\psi}
\]

$\lambda$ is called the eigenvalue of $\ket{\psi}$. Show that the normalized form of the state from part a) is an eigenstate of the $H$ gate. What is the eigenvalue?
  \end{question}
  \tcblower{}
  \begin{proof}
    If we represent the state in part a as $\ket{\psi} = a(\ket{0} + \ket{+})$, where $a = \frac{1}{\sqrt{2 + \sqrt{2}}}$, then we get,
    \begin{equation}
      \begin{split}
        H\ket{\psi} &= H\left(a(\ket{0} + \ket{+})\right)\\
        &= a(H(\ket{0} + \ket{+}))\\
        &= a(\ket{+} + \ket{0}) = 1\cdot\ket{\psi}
      \end{split}
    \end{equation}
    Therefore, $\ket{\psi}$ is an eigenstate of $H$ with an eigenvalue of $1$.
  \end{proof}
\end{solution}

\begin{solution}{Part c}\label{ques:3c}
  \begin{question}
    What single-qubit states are reachable from $\ket{0}$ using only $H$ and $P$, i.e. via any sequence of $H$ and $P$ gates? Are there finitely or infinitely many? Characterize all of the reachable states, up to global phase.
\\ Hint: Start by just playing with applying different sequences of matrices to the $\ket{0}$ state and look for a pattern.
  \end{question}
  \tcblower{}
  \begin{proof}
    There are a finitely many states that can be reached from $\ket{0}$ using only $H$ and $P$. We first show the following,
    \begin{equation}
      PH\ket{0} = P\ket{+} = \frac{1}{\sqrt{2}}H(\ket{0} + i\ket{1}) = \ket{i}
      \label{eq:ph}
    \end{equation}
    Similarly, we have,
    \begin{equation}
      P^2H\ket{0} = P^2\ket{+} = \ket{-}
      \label{eq:p2h}
    \end{equation}
    Also,
    \begin{equation}
      P^3H\ket{0} = P^3\ket{+} = \ket{-i}
      \label{eq:p3h}
    \end{equation}
    We also state the following from Equations~\ref{eq:ph},~\ref{eq:p2h},~\ref{eq:p3h},
    \begin{equation}
      \begin{split}
        HPH\ket{0} &= \ket{-i}\\
        HP^2H\ket{0} &= \ket{1}\\
        HP^3H\ket{0} &= \ket{i}
      \end{split}
      \label{eq:hpih}
    \end{equation}

    Also note that $P^4 = I$ and $H^2 = I$. Therefore, a non-trivial application of a unitary made up of a sequence of $H$ and $P$ gates will be of the form $H^{h}\cdot P^{p_n}\cdot H\cdot P^{p_{n-1}}\cdot H\cdot P^{p_{n-2}}\cdots H\cdot P^{p_1}\cdot H$ where $h\in \{0, 1\}\ \wedge\ \forall i:\ p_i \in \{1, 2, 3\}$. In words, we start with a Hadamard gate and we alternate between phase gates and Hadamards in between and may or may not end with a Hadamard gate. Note that starting with a phase gate has no effect on the $\ket{0}$ state and hence we can ignore its effect. Therefore, the only set of states that are reachable by such a unitary belong to the set $\mathbb{S} = \{\ket{0}, \ket{1}, \ket{+}, \ket{-}, \ket{i}, \ket{-i}\}$. We can easily prove this by induction on $i$.
  \end{proof}
\end{solution}
