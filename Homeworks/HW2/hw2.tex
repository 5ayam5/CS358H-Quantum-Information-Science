\documentclass[11pt]{article}
\usepackage[letterpaper, margin=2cm]{geometry}
\usepackage{titlesec}
\usepackage{mdframed}
\usepackage[dvipsnames]{xcolor} % for color names, must be loaded before tikz
\usepackage{ifthen}
\usepackage{comment}
\usepackage{fancyhdr}
\usepackage{titling}
\usepackage{hyperref}
\usepackage{enumitem}
\usepackage{tikz}
\usepackage{amsmath, amssymb, amsthm}
\usepackage{mathtools}
\usepackage[braket, qm]{qcircuit}

\providecommand{\due}{Due Wednesday, September 15 at 11:59 PM}
\lhead{CS358H, M375T, ES377} \rhead{}
\lfoot{\due} \cfoot{} \rfoot{Page \thepage}
\renewcommand{\footrulewidth}{0.4pt}
\pagestyle{fancy}

% Eliminates the spacing in the title that remains from the empty author section.
\preauthor{}
\postauthor{}

\titleformat{\section}[runin]{\large\bfseries}{\thesection .}{3pt}{}
\titleformat{\subsection}[runin]{\bfseries}{\thesubsection)}{3pt}{}
\renewcommand\thesubsection{\alph{subsection}}

% Defines the solution environment. Toggle solutions between true and false to either show or hide solutions. Also, the solution environment takes an optional argument of arbitrary text to be inserted in the solution header.
\newboolean{solutions}
\setboolean{solutions}{false}
\ifthenelse{\boolean{solutions}}
{\newenvironment{solution}{\begin{mdframed}[skipbelow=0pt, linecolor=White, backgroundcolor=Green!10]\textbf{Solution:}}{\end{mdframed}}}
{\excludecomment{solution}}

\allowdisplaybreaks

\DeclareMathOperator{\CNOT}{CNOT}

\begin{document}

\title{Introduction to Quantum Information Science\\Homework 2}
\date{\due}

\maketitle

\section{More fun with matrices}

\subsection{[1 Point]} Give an example of a $2 \times 2$ unitary matrix where the diagonal entries are 0 but all of the off-diagonal entries are nonzero.


\subsection{[2 Points]} Give an example of a $4 \times 4$ unitary matrix satisfying the same condition.

\subsection{[2 Points]} Is it possible to have a $3 \times 3$ unitary matrix with this condition? If so, give an example. If not, prove it!


\section{Single Qubit Quantum Circuits} For the following circuits, calculate the output state before the measurement. Then calculate the measurement probabilities in the specified basis. Here we use:

\begin{gather*}
X = \begin{bmatrix} 0 & 1 \\ 1 & 0 \end{bmatrix},\quad Y = \begin{bmatrix} 0 & -i \\ i & 0 \end{bmatrix},\quad Z = \begin{bmatrix} 1 & 0 \\ 0 & -1 \end{bmatrix}\\
H = \frac{1}{\sqrt{2}} \begin{bmatrix} 1 & 1 \\ 1 & -1 \end{bmatrix},\quad R_{\pi/4} = \frac{1}{\sqrt{2}} \begin{bmatrix} 1 & -1 \\ 1 & 1 \end{bmatrix},\quad P = \begin{bmatrix} 1 & 0 \\ 0 & i \end{bmatrix},\quad T = \begin{bmatrix} 1 & 0 \\ 0 & e^{i\pi/4} \end{bmatrix}
\end{gather*}

\subsection{[2.5 Points]} Measure in the $\{\ket{0}, \ket{1}\}$-basis:
\[
\Qcircuit @C=1em @R=.7em {
  \lstick{ \ket{0}}   & \gate{H} &  \gate{Z} & \gate{H} & \gate{P} & \meter
}
\]

\subsection{[2.5 Points]} Measure in the $\{\ket{+}, \ket{-}\}$-basis:
\[
\Qcircuit @C=1em @R=.7em {
    \lstick{\ket{0}} & \gate{R_{\pi/4}} &  \gate{Z} & \gate{Y} & \gate{H} & \meter
}
\]

\subsection{[2.5 Points]} Measure in the $\{\ket{+}, \ket{-}\}$ basis:
\[
\Qcircuit @C=1em @R=.7em {
    \lstick{ \ket{+}} & \gate{T} &  \gate{H} & \meter
}
\]

\subsection{[2.5 Points]} Measure in the $\{\ket{i}, \ket{-i}\}$-basis:
\[
\Qcircuit @C=1em @R=.7em {
    \lstick{ \ket{+}} & \gate{T} &  \gate{Z} & \gate{T} & \meter
}
\]


\section{Miscellaneous}

\subsection{[1 Point]} Normalize the state $\ket{0} + \ket{+}$.


\subsection{[1 Point]} We say a quantum state vector $\ket{\psi}$ is an eigenvector or eigenstate of a matrix $\Lambda$ if the following equation holds for some number $\lambda$:
\[
\Lambda \ket{\psi} = \lambda \ket{\psi}
\]

$\lambda$ is called the eigenvalue of $\ket{\psi}$. Show that the normalized form of the state from part a) is an eigenstate of the $H$ gate. What is the eigenvalue?


\subsection{[6 Points]} What single-qubit states are reachable from $\ket{0}$ using only $H$ and $P$, i.e. via any sequence of $H$ and $P$ gates? Are there finitely or infinitely many? Characterize all of the reachable states, up to global phase.
\\ Hint: Start by just playing with applying different sequences of matrices to the $\ket{0}$ state and look for a pattern.


\section{Distinguishability of states} Say you are given a state $\ket{\psi}$ that is either $\ket{0}$ or $\ket{1}$ but you don't know which. You can distinguish the two via a measurement in the $\{\ket{0}, \ket{1}\}$-basis.

\subsection{[6 Points]} But what if $\ket{\psi}$ is either $\ket{0}$ or $\ket{+}$ (with equal probability)? Give the protocol that distinguishes the two states with with a failure probability of $\sin^2(\frac{\pi}{8})\approx .146$. Show explicitly that your protocol achieves this failure probability.
\\ Note that when we ask for a protocol, we mean some step-by-step algorithm that ends by outputting ``I think this was $\ket{0}$'' or ``I think this was $\ket{+}$''.
\\ Hint: Read Section 5.2 of the textbook.


\subsection{[Extra Credit, 5 Points]} Prove that this is optimal.


\subsection{[2 Points]} What is the failure probability if you measure in the $\{\ket{0}, \ket{1}\}$ basis?



\end{document}

